\documentclass{article}
\usepackage{parskip}
\usepackage{ngerman}
\usepackage[utf8]{inputenc}
\usepackage{fullpage}
\usepackage{qtree}
\usepackage{blindtext}
\setlength{\parskip}{0.7cm}

\begin{document}
\section{Task}
\par 1. Retrieve the title and the creator name of all artifacts with value over 10000 Euros.
$$Answer(name, title, id) := \pi_{name, id}(Artists) \bowtie \rho_{artistId \rightarrow id}(\pi_{title, artistId}(\sigma_{value>1000}(Artifacts)))$$
\par 2. Name all exhibitions where paintings from the ‘Dutch painters’ collection were exhibited.
$$Art(id) := \pi_{id}(\sigma_{collectionTitle = 'Dutch Paintings'}(Artifacts)) \cap \pi_{id}(ArtifactsPaintings)$$
$$Answer(exhibitionTitle) := \pi_{exhibitionTitle}(ExhibitedAt\bowtie Art)$$
\par 3. Name artists who influenced others but had no influences.
$$Influencer(id) := \rho_{influencerId \rightarrow id}(\pi_{influencerId}(influencedBy)$$
$$Influencee(id) := \rho_{influenceeId \rightarrow id}(\pi_{influenceeId}(influencedBy))$$
$$Answer(name) := \pi_{name}((Influencer \setminus Influencee)\bowtie (Artists)))$$
\par 4. Which newspapers have advertised exhibitions where paintings in oil canvas and marble sculptures were shown. \newline 
We assume the task is to display exhibitions where both Artifact types are advertised together.
$$Oil(artifactsId) := \rho_{id \rightarrow artifactsId}(\pi_{id}(\sigma_{canvas = 'oil'}(ArtifactsPaintings)))$$
$$Marble(artifactsId) := \rho_{id \rightarrow artifactsId}(\pi_{id}(\sigma_{material = 'marble'}(ArtifactsSculptures)))$$
$$Ex(exhibitionTitle) := \pi_{exhibitionTitle}(ExhibitedAt \bowtie Oil) \cap \pi_{exhibitionTitle}(ExhibitedAt \bowtie Marble)$$
$$News(mediumName) := \rho_{name \rightarrow mediumName}(\pi_{name}(\sigma_{type = 'newspaper'}(Media)))$$
$$Answer(mediumName) := \pi_{mediumName}(Advertisements \bowtie Ex) \cap News$$

\newpage
\section{Task}
\par a) Retrieve the artifactsname and exhibitionstitle of all artifacts with value over 1000 Euro that were part of an exhibition advertised on the 'Daily Planet'.
\par b) 
\Tree   [ 
            .$\pi_{title, exhibitionTitle}$ 
            [   
                .$\sigma_{value>1000 \land mediumName='DailyPlanet'}$ 
                [
                    .$\bowtie$
                    Artifacts
                    [
                        .$\bowtie$
                        [
                            .$\rho_{ArtifactId \rightarrow id}$
                            ExhibitedAt
                        ]
                        Advertisements
                    ]
                ]
            ]
        ]
\par c) 
start expression
$$\pi_{title, exhibitionTitle}(\sigma_{value>1000 \land mediumname='DailyPlanet'}(Artifacts \bowtie (\rho_{artifactId \leftarrow id }(ExhibitedAt) \bowtie Advertisements))$$
splitting sigma
$$\pi_{title, exhibitionTitle}(\sigma_{value>1000}(\sigma_{mediumname='DailyPlanet'}((Artifacts \bowtie (\rho_{artifactId \leftarrow id }(ExhibitedAt) \bowtie Advertisements)))$$
push selection down
$$\pi_{title, exhibitionTitle}(\sigma_{value>1000}(Artifacts) \bowtie (\rho_{artifactId \leftarrow id }(ExhibitedAt) \bowtie \sigma_{mediumname='DailyPlanet'} (Advertisements)))$$
order joins by size:
$$\pi_{title, exhibitionTitle}((\sigma_{value>1000}(Artifacts) \bowtie \rho_{artifactId \leftarrow id }(ExhibitedAt)) \bowtie \sigma_{mediumname='DailyPlanet'}(Advertisements))$$
cannot push projections down, because the join needs the other attributes and after the join we would have to project again

\Tree   [ 
            .$\pi_{title, exhibitionTitle}$ 
            [   
                .$\bowtie$
                [
                    .$\bowtie$
                    [
                        .$\sigma_{value>1000}$ 
                        Artifacts
                    ]
                    [
                        .$\rho_{ArtifactId \rightarrow id}$
                        ExhibitedAt
                    ]
                ]
                [
                    .$\sigma_{mediumName='DailyPlanet'}$
                    Advertisements
                ]
            ]
        ]
\end{document}
